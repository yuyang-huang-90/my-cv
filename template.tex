%% start of file `template.tex'.
%% Copyright 2006-2013 Xavier Danaux (xdanaux@gmail.com).
%
% This work may be distributed and/or modified under the
% conditions of the LaTeX Project Public License version 1.3c,
% available at http://www.latex-project.org/lppl/.


\documentclass[10pt,a4paper,roman]{moderncv}        % possible options include font size ('10pt', '11pt' and '12pt'), paper size ('a4paper', 'letterpaper', 'a5paper', 'legalpaper', 'executivepaper' and 'landscape') and font family ('sans' and 'roman')

% moderncv themes
\moderncvstyle{banking}                             % style options are 'casual' (default), 'classic', 'oldstyle' and 'banking'
\moderncvcolor{black}                               % color options 'blue' (default), 'orange', 'green', 'red', 'purple', 'grey' and 'black'
%\renewcommand{\familydefault}{\sfdefault}         % to set the default font; use '\sfdefault' for the default sans serif font, '\rmdefault' for the default roman one, or any tex font name
%\nopagenumbers{}                                  % uncomment to suppress automatic page numbering for CVs longer than one page

% character encoding
%\usepackage[utf8]{inputenc}                       % if you are not using xelatex ou lualatex, replace by the encoding you are using
%\usepackage{CJKutf8}                              % if you need to use CJK to typeset your resume in Chinese, Japanese or Korean

% adjust the page margins
\usepackage[scale=0.75]{geometry}
\usepackage{fullpage}

%\setlength{\hintscolumnwidth}{3cm}                % if you want to change the width of the column with the dates
%\setlength{\makecvtitlenamewidth}{10cm}           % for the 'classic' style, if you want to force the width allocated to your name and avoid line breaks. be careful though, the length is normally calculated to avoid any overlap with your personal info; use this at your own typographical risks...

% personal data
\name{Uyo(Yuyang)}{Ko(Huang)}
%\title{Resumé title}                               % optional, remove / comment the line if not wanted
%\address{RM 604}{Kuramae 4-35-13}{Taito-ku, Tokyo, Japan, 111-0051}% optional, remove / comment the line if not wanted; the "postcode city" and "country" arguments can be omitted or provided empty
\phone[mobile]{+818043399496}                   % optional, remove / comment the line if not wanted; the optional "type" of the phone can be "mobile" (default), "fixed" or "fax"
%\phone[fixed]{+2~(345)~678~901}
%\phone[fax]{+3~(456)~789~012}
\email{sigefriedhyy@gmail.com}                               % optional, remove / comment the line if not wanted
%\homepage{kdb-m.org}                         % optional, remove / comment the line if not wanted
\social[linkedin]{https://www.linkedin.com/in/ko-uyo-46360092/}                        % optional, remove / comment the line if not wanted
%\social[twitter]{jdoe}                             % optional, remove / comment the line if not wanted
\social[github]{https://github.com/sigefried}                              % optional, remove / comment the line if not wanted
%\extrainfo{additional information}                 % optional, remove / comment the line if not wanted
\photo[64pt][0.4pt]{picture}                       % optional, remove / comment the line if not wanted; '64pt' is the height the picture must be resized to, 0.4pt is the thickness of the frame around it (put it to 0pt for no frame) and 'picture' is the name of the picture file
%\quote{Some quote}                                 % optional, remove / comment the line if not wanted

% to show numerical labels in the bibliography (default is to show no labels); only useful if you make citations in your resume
%\makeatletter
%\renewcommand*{\bibliographyitemlabel}{\@biblabel{\arabic{enumiv}}}
%\makeatother
%\renewcommand*{\bibliographyitemlabel}{[\arabic{enumiv}]}% CONSIDER REPLACING THE ABOVE BY THIS

% bibliography with mutiple entries
%\usepackage{multibib}
%\newcites{book,misc}{{Books},{Others}}
%----------------------------------------------------------------------------------
%            content
%----------------------------------------------------------------------------------
\begin{document}
%\begin{CJK*}{UTF8}{gbsn}                          % to typeset your resume in Chinese using CJK
%-----       resume       ---------------------------------------------------------
\makecvtitle


\section{Work Experience}
\cventry{November 2021--Present}{Software Engineer}{Indeed}{Tokyo Japan}{}{
  \begin{itemize}
    \item Work in the  backend team for the candidate matching system (team of approx. 12).
    \item Designed and developed the candidate recommendation system for employer, supporting 8 different countries.
      \begin{itemize}
        \item Worked on the inference server to improve the scalability by
          supporting 3x the numbers of candidates from the storage perspective.
        \item Worked on the job feature service to improve the stability of the service and make it scale to support 14x more traffic volume.
      \end{itemize}
  \end{itemize}
}

\cventry{August 2020--November 2021}{Software Engineer}{Google}{Munich Germany}{}{
  \begin{itemize}
    \item Worked in Munich Android Auto Development Team (team of approx. 8).
    \item Designed and developed vehicle-to-phone connectivity solutions.
      \begin{itemize}
        \item Built next generation vehicle-to-phone connectivity system
          software.
        \item Built next generation projecting solution for Android phones.
      \end{itemize}
  \end{itemize}
}

\cventry{February 2018--August 2020}{Software Engineer, Associate}{Goldman Sachs}{Tokyo Japan}{}{
  \begin{itemize}
    \item Worked in the Equity Engineering Group (team of approx. 8).
    \item Improved system latency and scalability.
    \item Main projects:
      \begin{itemize}
        \item Built key components for a next generation sequencer based ultra-low latency trading platform.
        \item Worked as an site reliability engineer to provide L3 support for the electronic trading platform.
      \end{itemize}
  \end{itemize}
}

\cventry{April 2016--February 2018}{Linux System Research and Development Engineer}{Sony}{Tokyo
Japan}{}{
  \begin{itemize}
    \item Worked in the Base System R\&D Department, Linux Kernel
      R\&D Section (team of approx. 6).
    \item Held a team member position at the AI/Robotics Business Unit, System
      Software Development Section (team of approx. 20).
    \item Performed parallel work on two main projects
      \begin{itemize}
        \item Designed and developed a secure application framework for embedding Linux in next-generation Internet of Things devices and robots.
        \item Linux kernel/driver development for both current and
          next-generation embedded system platform.
      \end{itemize}
  \end{itemize}
}

\section{Education}
\cventry{April 2014--March 2016}{M.S. in Information and Communication Engineering}{The University of Tokyo}{Tokyo
Japan}{}{Graduate {S}chool of Information Science and Technology}
\cventry{September 2009--July 2013}{B.Eng. in Electrical Engineering and Automation}{Dong Hua University}{Shanghai China}{}{Department of Electrical Engineering}  % arguments 3 to 6 can be left empty

\section{Selected Projects}

\cventry{November 2021--Present}{Candidate matching system backend
  development supporting 8 countries.}{Candidate matching system backend development.}{Java, Kotlin}{}{\textbf{Techniques:} Performance analysis and
distributed system development.}
\begin{itemize}
  \item[--]{Integrated the inference server with Amazon DynamoDB as permanent
    backend storage which increased the supported candidate count by 3x from the storage perspective.}
  \item[--]{Did performance analysis on the job feature service. Applied rate
    limiter and multi-layer caching which resolved the performance bottleneck
  and supported 14x more traffic volume.}
\end{itemize}

\cventry{August 2020--November 2021}{Next-generation phone to vehicle
connectivity solution.}{Android Auto and AAOS software
  development}{C/C++,Java}{}{\textbf{Techniques:}  Android development,
  performance analysis, system service development.}
\begin{itemize}
  \item[--]{Designed and implemented next generation connectivity system
    software which provided a unified phone to vehicle communication layer.
    This software manages USB, Wi-Fi and Bluetooth(RFComm, BLE) as underline transports
    and makes low level connection details agnostic to the application used.}
  \item[--]{Design and implement an application level projecting solution for phone to vehicle projection.}
\end{itemize}

\cventry{February 2018--July 2020}{Next-generation ultra-low latency
  trading platform.}{Electronic trading
platform development}{Java,C/C++,Slang,Python}{}{\textbf{Techniques:}  Performance
analysis, algorithm design and implementation and distributed system development}
\begin{itemize}
  \item[--]{Designed and implemented a next generation sequencer based ultra-low
    latency electrical trading platform, which provided less than 150 micro
  second end to end latency for synthetic market access.}
  \item[--]{Provided L3 support for the platform.}
\end{itemize}

\cventry{April 2016--February 2018}{Linux kernel and security software
development for next-generation platform.}{Linux kernel and system security
software development.}{C/C++,Python,Golang}{}{\textbf{Techniques:}  Embedded
system development, Linux kernel development and containerization}
\begin{itemize}
  \item[--]{Responsible for Linux kernel and driver development for next
    generation platform. }
  \item[--]{Reduced the kernel crash rate by around 30\% and reduced the kernel boot time by around 40\%.}
  \item[--]{Designed and implemented containerization software for embedded Linux
    platform with limited resources.}
  \item[--]{Coordinated container software functioning with other system middleware.}
\end{itemize}

\cventry{April 2014--March 2016}{Development of highly accurate pedestrian
navigation system for urban canyon environment}{Height-Aided
PNS}{C/C++,Python,Java}{}{\textbf{Techniques:} Optimization, self-localization,
GNSS, Wi-Fi localization and Android programming}
\begin{itemize}
  \item[--]{Designed, implemented, and evaluated a height aided GNSS algorithm for
    pedestrian navigation in an urban environment under the supervision of a senior researcher
  and professor. This method reduced the mean error in GNSS localization from 17 meters to 12 meters in an urban canyon.}
  \item[--]{Integrated the height aided GNSS with PDR and Wi-Fi localization
    system. The integrated pedestrian navigation system could achieve accuracy
  with around 6.5 meters mean error in the urban canyon.}
  \item[--]{This project was my master's thesis. The output of this project was sold to a well-known company.}
\end{itemize}

\section{Skills}
\cvlistitem{Algorithm design, analysis and implementation.}
\cvlistitem{Distributed system design, implementation.}
\cvlistitem{Linux kernel development, system software development, embedded platform development.}
\cvlistitem{Linux system administration.}
\cvlistitem{Android system development.}
\cvlistitem{Android application development.}
\cvlistitem{In-depth experience and knowledge of Linux security mechanism: discretionary access control, capabilities, namespace, seccomp, cgroups.}
\cvlistitem{In-depth experience and knowledge of container software: runC and Docker.}
\cvlistitem{In-depth experience and knowledge of networking stack development.}
\cvlistitem{In-depth experience and knowledge of Open Source software: Spring, Kafka, Elastic Search, Hadoop, Spark.}
\cvlistitem{Experience using AWS and Google Cloud.}
\cvlistitem{Programming languages: C/C++, Java, Kotlin, Python, Assembly, Golang, Ruby/Rails.}
\cvlistitem{Languages: Chinese(native), English(fluent), Japanese(fluent),
German(beginner)}

\nocite{*}
\bibliographystyle{ieeetr}
\bibliography{publications}

\vspace{5mm}
%%
%\noindent\rule{\textwidth}{0.4pt}
%\textbf{The answer to the survey question:  B}

%\section{Languages}
%\cvitemwithcomment{Japanese}{NJLPT N1}{}
%\cvitemwithcomment{English}{TOEFL(ibt) 94, GRE 314}{}
%\cvitemwithcomment{Chinese}{Native Speaker}{}
%\cvitemwithcomment{French}{Begineer's Level}{}

\end{document}


%% end of file `template.tex'.
