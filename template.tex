%% start of file `template.tex'.
%% Copyright 2006-2013 Xavier Danaux (xdanaux@gmail.com).
%
% This work may be distributed and/or modified under the
% conditions of the LaTeX Project Public License version 1.3c,
% available at http://www.latex-project.org/lppl/.


\documentclass[10pt,a4paper,roman]{moderncv}        % possible options include font size ('10pt', '11pt' and '12pt'), paper size ('a4paper', 'letterpaper', 'a5paper', 'legalpaper', 'executivepaper' and 'landscape') and font family ('sans' and 'roman')

% moderncv themes
\moderncvstyle{banking}                             % style options are 'casual' (default), 'classic', 'oldstyle' and 'banking'
\moderncvcolor{black}                               % color options 'blue' (default), 'orange', 'green', 'red', 'purple', 'grey' and 'black'
%\renewcommand{\familydefault}{\sfdefault}         % to set the default font; use '\sfdefault' for the default sans serif font, '\rmdefault' for the default roman one, or any tex font name
%\nopagenumbers{}                                  % uncomment to suppress automatic page numbering for CVs longer than one page

% character encoding
%\usepackage[utf8]{inputenc}                       % if you are not using xelatex ou lualatex, replace by the encoding you are using
%\usepackage{CJKutf8}                              % if you need to use CJK to typeset your resume in Chinese, Japanese or Korean

% adjust the page margins
\usepackage[scale=0.75]{geometry}
\usepackage{fullpage}

%\setlength{\hintscolumnwidth}{3cm}                % if you want to change the width of the column with the dates
%\setlength{\makecvtitlenamewidth}{10cm}           % for the 'classic' style, if you want to force the width allocated to your name and avoid line breaks. be careful though, the length is normally calculated to avoid any overlap with your personal info; use this at your own typographical risks...

% personal data
\name{Uyo(Yuyang)}{Ko(Huang)}
%\title{Resumé title}                               % optional, remove / comment the line if not wanted
\address{RM 604}{Kuramae 4-35-13}{Taito-ku, Tokyo, Japan, 111-0051}% optional, remove / comment the line if not wanted; the "postcode city" and "country" arguments can be omitted or provided empty
\phone[mobile]{+818043399496}                   % optional, remove / comment the line if not wanted; the optional "type" of the phone can be "mobile" (default), "fixed" or "fax"
%\phone[fixed]{+2~(345)~678~901}
%\phone[fax]{+3~(456)~789~012}
\email{sigefriedhyy@gmail.com}                               % optional, remove / comment the line if not wanted
%\homepage{kdb-m.org}                         % optional, remove / comment the line if not wanted
\social[linkedin]{https://www.linkedin.com/in/ko-uyo-46360092/}                        % optional, remove / comment the line if not wanted
%\social[twitter]{jdoe}                             % optional, remove / comment the line if not wanted
%\social[github]{https://github.com/sigefried}                              % optional, remove / comment the line if not wanted
%\extrainfo{additional information}                 % optional, remove / comment the line if not wanted
\photo[64pt][0.4pt]{picture}                       % optional, remove / comment the line if not wanted; '64pt' is the height the picture must be resized to, 0.4pt is the thickness of the frame around it (put it to 0pt for no frame) and 'picture' is the name of the picture file
%\quote{Some quote}                                 % optional, remove / comment the line if not wanted

% to show numerical labels in the bibliography (default is to show no labels); only useful if you make citations in your resume
%\makeatletter
%\renewcommand*{\bibliographyitemlabel}{\@biblabel{\arabic{enumiv}}}
%\makeatother
%\renewcommand*{\bibliographyitemlabel}{[\arabic{enumiv}]}% CONSIDER REPLACING THE ABOVE BY THIS

% bibliography with mutiple entries
%\usepackage{multibib}
%\newcites{book,misc}{{Books},{Others}}
%----------------------------------------------------------------------------------
%            content
%----------------------------------------------------------------------------------
\begin{document}
%\begin{CJK*}{UTF8}{gbsn}                          % to typeset your resume in Chinese using CJK
%-----       resume       ---------------------------------------------------------
\makecvtitle

\section{Objective}
Seeking full time senior software engineer position(6 YOE).

\section{Work Experience}
\cventry{November 2021--Present}{Software Engineer}{Indeed}{Tokyo
Japan}{Recommendation backend team}{
  \begin{itemize}
    \item Designed and implemented the candidate recommendation system for
      specific jobs across 8 different countries.
    \item Led a sub-team to improve multiple backend services from scalability and stability perspective.
  \end{itemize}
}

\cventry{August 2020--November 2021}{Software Engineer}{Google}{Munich Germany}{Android auto team}{
  \begin{itemize}
    \item Designed and implemented a next generation phone to vehicle connectivity library.
    \item Designed and implemented a next generation projecting library for Android phones.
    \item Led a sub-team to redesign the threading model of phone to car communication library.
  \end{itemize}
}

\cventry{February 2018--August 2020}{Software Engineer}{Goldman Sachs}{Tokyo Japan}{Equity trading platform development team}{
  \begin{itemize}
    \item Designed and implemented key components for a next generation sequencer based ultra-low latency trading platform.
    \item Worked as a site reliability engineer to provide L3 support for the electronic trading platform.
    \item Led a sub-team to improve the supportability of the newly built trading platform.
  \end{itemize}
}

\cventry{April 2016--February 2018}{Software Engineer}{Sony}{Tokyo
Japan}{Base system research and development team}{
  \begin{itemize}
    \item Designed and implemented a secure application framework for embedding Linux in the Sony robot dog(Aibo).
    \item Linux kernel/driver development and performance turning for the Sony robot dog(Aibo).
  \end{itemize}
}

\section{Selected Projects}

\cventry{November 2021--Present}{Recommends best matched candidates for specific open job positions.}{Candidate recommendation system backend development, @Indeed}{Java, Kotlin}{}{
  \begin{itemize}
    \item Integrated the inference server with Amazon DynamoDB as a permanent backend storage, which increased the supported candidate count by 3x from the storage perspective.
    \item Introduced rate limiter and multi-layer caching to the job feature service, which enlarged the supported traffic volume by 14x.
    \item Migrated gRPC and HTTPs based legacy APIs to GraphQL based APIs to allow client to customize their queries which further reduced 50\% of unnecessary downstream queries.
  \end{itemize}
}

\cventry{August 2020--November 2021}{Next-generation phone to vehicle connectivity and projection library.}{Android Auto and AAOS software development, @Google}{C/C++, Android Java}{}{
  \begin{itemize}
    \item Designed and implemented a next generation phone to vehicle connectivity library which provided a unified communication layer. This library manages USB, Wi-Fi and Bluetooth (RFComm, BLE) as underlying transports and makes low level connection details agnostic to the application layer.
    \item Improved the connectivity library stability and extensibility by redesigning the threading model. 
    \item Designed and implemented an application-level projecting library which supports phone to vehicle projection for android phones.
  \end{itemize}
}

\cventry{February 2018--July 2020}{Next-generation ultra-low latency trading platform.}{Electronic trading
platform development, @Goldman Sachs}{Java,C/C++,Python}{}{
  \begin{itemize}
    \item Designed and implemented a next generation sequencer based ultra-low latency electrical trading platform, which provided less than 150 microsecond end-to-end latency for synthetic market access.
    \item Improved the supportability of the trading platform by introducing Elasticsearch based log collecting/analyzing service.
    \item Provided L3 support for the platform as a site reliability engineer.
  \end{itemize}
}

\cventry{April 2016--February 2018}{Linux kernel and security software development for Sony robot dog.}{Linux kernel and system security
software development, @Sony}{C/C++,Python,Go}{}{
  \begin{itemize}
    \item Customized the Linux kernel and driver for the Sony robot dog(Aibo).
    \item Reduced the kernel crash rate by around 30\% and improved the kernel boot time by around 40\%.
    \item Designed and implemented a runC based containerization software for embedded Linux platforms with limited resources.
  \end{itemize}
}

\section{Skills}
\cvlistitem{\textbf{Programming language:} C/C++, Java, Kotlin, Python, Android Java/Kotlin, Go}
\cvlistitem{\textbf{Cloud platform:} AWS, Google Cloud.}
\cvlistitem{\textbf{Distribute system:} Kafka, Elasticsearch, Hadoop, Spark, Redis.}
\cvlistitem{\textbf{Database:} MySQL, DynamoDB.}
\cvlistitem{\textbf{Embedded system:} Android system, Linux kernel, Linux system.}
\cvlistitem{\textbf{Containerization:} runC/Docker, Linux namespace, seccomp, cgroups.}
\cvlistitem{\textbf{Java Framework:} Spring MVC, Spring Boot, Netflix DGS.}
\cvlistitem{\textbf{Other technologies:} gRPC, GraphQL, Rate limiter, Zero GC low latency Java Development.}
\cvlistitem{\textbf{Languages:} Chinese(native), English(fluent), Japanese(fluent),
German(beginner)}

\section{Education}
\cventry{April 2014--March 2016}{M.S. in Information and Communication Engineering}{The University of Tokyo}{Tokyo
Japan}{}{Graduate {S}chool of Information Science and Technology}
\cventry{September 2009--July 2013}{B.Eng. in Electrical Engineering and Automation}{Dong Hua University}{Shanghai China}{}{Department of Electrical Engineering}  % arguments 3 to 6 can be left empty


% \nocite{*}
% \bibliographystyle{ieeetr}
% \bibliography{publications}

% \vspace{5mm}
%%
%\noindent\rule{\textwidth}{0.4pt}
%\textbf{The answer to the survey question:  B}

%\section{Languages}
%\cvitemwithcomment{Japanese}{NJLPT N1}{}
%\cvitemwithcomment{English}{TOEFL(ibt) 94, GRE 314}{}
%\cvitemwithcomment{Chinese}{Native Speaker}{}
%\cvitemwithcomment{French}{Begineer's Level}{}

\end{document}


%% end of file `template.tex'.
